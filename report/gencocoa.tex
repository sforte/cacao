\section{Generalized CoCoA}

\subsection{Setup}
As presented in the CoCoA paper \cite{CoCoA} the algorithm currently has support for problems having the following
primal form:
\sdcaPrimal

It is though possible, in a very similar theoretical and implementative framework, to solve problems of the following
more general form:
\gensdcaPrimal
where $\primalreg$ is a 1-convex function with respect to the L2 norm.

Similarly as for the L2 regularized problem solved by CoCoA, and as done in \cite{ShalevShwartz:2012tn}, we can define
a dual problem:
\gensdcaDual
where:
\vwdef
A primal-dual relation also holds for these two problems, with the duality gap \dualitygap acting as an upperbound
to the suboptimality of the primal and the dual.

\subsection{Method description}
In this section we present a generalized CoCoA algorithm to solve this more general problem. The high level structure of
the algorithm is very similar (if not almost identical) to the original algorithm and it is as follows: \\
\gencocoa
The only difference to the original CoCoA is the use of the $\vv(\av)$ vectors instead of the $\wv(\av)$. This vector
is computed at the end, using the gradient of the dual conjugate of the regularizer. Another difference is that the
local method is now required to optimize the generalized dual form as in \ref{eq:gensdcaDual} (for its local coordinated)
instead of the original CoCoA problem.
Same as in \cite{CoCoA} we'll also make the following assumption on the local solver:
\localgeomimpr
where:
\suboptimality
Given this we can state an theorem equivalent to Theorem 2 in \cite{CoCoA} (the proof can be found in the appendix):
\convergence

\subsection{Local SDCA}
In this section we'll present an SDCA method, very similar to the one in \cite{ShalevShwartz:2012tn} used to solve the
generalized local dual problem. The algorithm is as follows and it's also very similar to the CoCoA one: \\
\genlocalsdca

Unfortunately solving the problem:
$$\schardproblem$$
can be very hard for complex regularizers and it's replaced in \cite{ShalevShwartz:2012tn} by the following relaxed
lowerbound problem (which still allows for finding the exact solution):
$$\screlaxedproblem$$
that is much simpler to solve since it only requires us to be able to compute the gradient $\nabla \dualreg$ and after
that it can easily be solved by line search. It is also possible to notice that this lowerbound problem is actually
strict in the case of L2 regularization.

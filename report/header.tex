%%%%%%%%%%%%%%%%%%%%%%%%%
%%%% Definitions, Operators, Commands %%%%
%%%%%%%%%%%%%%%%%%%%%%%%%

% ============ GENERAL ======================
\newcommand{\conj}[1]{\overline{#1}}
\newcommand{\ct}{\tilde{c}}
\newcommand{\dotp}[2]{\langle {#1},\, {#2} \rangle}
\newcommand{\mybf}{\color{cdarkblue}}
\newcommand{\Xc}{\mathcal{X}}
\newcommand{\Yc}{\mathcal{Y}}
\newcommand{\Nc}{\mathcal{N}}
\newcommand{\inv}[1]{#1^{-1}}
\newcommand{\vsp}{\vspace*{8pt}}
\newcommand{\ip}[2]{\langle {#1},\, {#2} \rangle}
\newcommand{\fromto}[3]{\sml{$#1 \le #2 \le #3$}}

% =========== OPTIMIZATION ================
\newcommand{\opt}{\star}
\newcommand{\xopt}{x^{\opt}}
\newcommand{\yopt}{y^{\opt}}
\newcommand{\popt}{p^{\opt}}
\newcommand{\dopt}{\alpha^{\opt}}
% notation for iterates
\renewcommand{\k}[1]{ {#1}_{k}}
\newcommand{\kp}[1]{ {#1}_{k+1}}
\newcommand{\km}[1]{{#1}_{k-1}}
\newcommand{\xk}{\k{x}}
\newcommand{\xkp}{\kp{x}}
\newcommand{\xkm}{\km{x}}
\newcommand{\stepk}{\k{\step}}
\newcommand{\step}{t}

% =========== NORMS =====================
\newcommand{\pnorm}[2]{\| {#1} \|_{#2}}
\newcommand{\norm}[1]{\pnorm{#1}{}}
\newcommand{\bignorm}[2]{\left\| {#1} \right\|_{#2}}
\newcommand{\norml}[1]{\pnorm{#1}{1}}
\newcommand{\bignorml}[1]{\bignorm{#1}{1}}
\newcommand{\infnorm}[1]{\pnorm{#1}{\infty}}
\newcommand{\biginfnorm}[1]{\bignorm{#1}{\infty}}
\newcommand{\oneinf}{\ell_{1,\infty}}
\newcommand{\onetwo}{\ell_{1,2}}
\newcommand{\oneinfnorm}[1]{\pnorm{#1}{1,\infty}}
\newcommand{\bigoneinf}[1]{\bignorm{#1}{1,\infty}}
\newcommand{\onetwonorm}[1]{\pnorm{#1}{1,2}}
\newcommand{\bigonetwo}[1]{\bignorm{#1}{1,2}}
\newcommand{\enorm}[1]{\pnorm{#1}{2}}
\newcommand{\bigenorm}[1]{\bignorm{#1}{2}}
\newcommand{\znorm}[1]{\pnorm{#1}{0}}
\newcommand{\bigznorm}[1]{\bignorm{#1}{0}}
\newcommand{\frob}[1]{\|{#1}\|_{\text{F}}}
\newcommand{\bigfrob}[1]{\bignorm{#1}{\text{F}}}
\newcommand{\grpnorm}[2]{\norm{#1}{\text{Gr}(#2)}}

% ============ INTEGER FUNCTIONS =============
\newcommand{\floor}[1]{\lfloor{#1}\rfloor}
\newcommand{\ceil}[1]{\lceil{#1}\rceil}
\newcommand{\bigfloor}[1]{\left\lfloor{#1}\right\rfloor}
\newcommand{\bigceil}[1]{\left\lceil{#1}\right\rceil}

% =========== SUMs, INTEGRALS, etc ==============
\newcommand{\sumton}{\sum_{i=1}^n}
\newcommand{\nlsum}{\sum\nolimits}
\newcommand{\nlprod}{\prod\nolimits}
\newcommand{\nlint}{\int\nolimits}
\newcommand{\nlmin}{\min\nolimits}
\newcommand{\nlmax}{\max\nolimits}

% =============== ANALYSIS =====================
\newcommand{\set}[1]{\left\{ {#1}\right\}}
\newcommand{\fdel}[2]{\frac{\partial {#1}}{\partial {#2}}}
\newcommand{\sdel}[2]{{\partial {#1}}/{\partial {#2}}}



% ================= SETS =======================
\newcommand{\R}{\mathbb{R}}
\newcommand{\C}{\mathbb{C}}
\newcommand{\integers}{\mathbb{Z}}
\newcommand{\posdef}[1]{S_{++}^{#1}}
\newcommand{\semidef}[1]{S_+^{#1}}
\newcommand{\1}{\mathbb{1}}


% =============== MISC CONSTANTS ===============
\newcommand{\half}{\tfrac{1}{2}}
\newcommand{\third}{\tfrac{1}{3}}
\newcommand{\fourth}{\tfrac{1}{4}}
\newcommand{\onehalf}{\tfrac{3}{2}}
\newcommand{\pfrac}[2]{\left(\tfrac{#1}{#2}\right)}


% ============== MATH KEYWORDS ================
\DeclareMathOperator*{\argmin}{argmin}
\DeclareMathOperator*{\argmax}{argmax}
\DeclareMathOperator{\dom}{dom}
\DeclareMathOperator{\interior}{int}
\DeclareMathOperator{\rank}{rank}
\DeclareMathOperator{\ri}{ri}
\DeclareMathOperator{\sgn}{sgn}
\DeclareMathOperator{\trace}{tr}
\DeclareMathOperator{\Diag}{Diag}
\DeclareMathOperator{\range}{range}
\DeclareMathOperator{\vect}{vec}
\DeclareMathOperator{\prox}{prox}
\DeclareMathOperator{\intr}{int}
\DeclareMathOperator*{\minimize}{minimize}

% distributed coordinate descent
\newcommand{\subopt}{{\varepsilon}_{\hspace{-0.08em}D}}
\newcommand{\suboptlocal}{{\varepsilon}_{\hspace{-0.08em}D,k}}

% ============= FROM MARTIN ===================
\DeclareMathOperator{\st}{\textrm{s.t.}}
\DeclareMathOperator*{\conv}{conv}
\DeclareMathOperator{\diam}{diam}
\DeclareMathOperator*{\E}{\mathbf{E}}
\DeclareMathOperator{\lmax}{\lambda_{\max}}
\DeclareMathOperator{\lmin}{\lambda_{\min}}
\DeclareMathOperator*{\supp}{supp}
\DeclareMathOperator*{\diag}{diag}
\DeclareMathOperator*{\sign}{sign}
\DeclareMathOperator*{\tr}{Tr}
\DeclareMathOperator*{\rk}{Rk}
\DeclareMathOperator*{\card}{card}
\DeclareMathOperator*{\nnz}{nnz}
\DeclareMathOperator*{\relint}{\mathop{relint}}

%norms
\providecommand{\abs}[1]{\left\lvert#1\right\rvert}
\providecommand{\norm}[1]{\left\lVert#1\right\rVert}
\providecommand{\dualnorm}[1]{\norm{#1}^*}
\providecommand{\frobnorm}[1]{\norm{#1}_{Fro}}
\providecommand{\tracenorm}[1]{\norm{#1}_{tr}}
\providecommand{\opnorm}[1]{\norm{#1}_{op}}
\providecommand{\maxnorm}[1]{\norm{#1}_{\max}}
\providecommand{\latGrNorm}[1]{\norm{#1}_{\groups}}

\newcommand{\ball}{\mathcal{B}}
\newcommand{\sphere}{\mathcal{S}}

\newcommand{\bigO}{O}
%\newcommand{\R}{\mathbb{R}}
%\newcommand{\X}{\mathcal{X}}
\newcommand{\Sym}{\mathbb{S}}

%Frank-Wolfe:
\newcommand{\domain}{\mathcal{D}}
\newcommand{\stepsize}{\gamma}
\newcommand{\Cf}{C_{\hspace{-0.08em}f}}
\newcommand{\x}{\bm{x}}
\newcommand{\y}{\bm{y}}
\newcommand{\z}{\bm{z}}
\newcommand{\s}{\bm{s}}
\newcommand{\uu}{\bm{u}}
\newcommand{\vv}{\bm{v}}
\newcommand{\wv}{{\bm{w}}}
\newcommand{\av}{\bm{\alpha}}
\newcommand{\bv}{\bm{b}}
\newcommand{\local}{{\textsf{\tiny local}}}
\newcommand{\other}{{\textsf{\tiny other}}}
\newcommand{\atoms}{\mathcal A}
\newcommand{\groups}{\mathcal G}
\newcommand{\row}{\text{row}}
\newcommand{\col}{\text{col}}
\newcommand{\lft}{\text{left}}
\newcommand{\rgt}{\text{right}}
%
\newcommand{\Spectahedron}{\mathcal{S}}
\newcommand{\SpectahedronLeOne}{\mathcal{S}_{\le 1}} %or _{\le 1}}
\newcommand{\SpectahedronLeT}{\mathcal{S}_{t}} %or _{\le t}}
\newcommand{\MaxCutPolytope}{\boxplus}
\newcommand{\FourPointPSD}{\Spectahedron^4_{\text{sparse}}}
\newcommand{\FourPointPSDplus}{\Spectahedron^{4+}_{\text{sparse}}}
\newcommand{\FourPointPSDminus}{\Spectahedron^{4-}_{\text{sparse}}}
\newcommand{\LOneBall}{\diamondsuit}
\newcommand{\signVec}{\mathbf{s}}
\newcommand{\N}{\mathbb{N}}
\newcommand{\id}{\mathbf{I}} % big i for identity
\newcommand{\ind}{\mathbf{1}} % indicator vectors
\newcommand{\0}{\mathbf{0}} % the origin
\newcommand{\unit}{\mathbf{e}} % unit basis vectors
\newcommand{\one}{\mathbf{1}} % all one vector
\newcommand{\zero}{\mathbf{0}}
\newcommand\SetOf[2]{\left\{#1\vphantom{#2}\right.\left|\vphantom{#1}\,#2\right\}}
\newcommand{\ignore}[1]{}%{\textbf{***begin ignore***}\\#1\textbf{***end ignore***}}


\newcommand{\todo}[1]{\marginpar[\hspace*{4.5em}\textbf{TODO}\hspace*{-4.5em}]{\textbf{TODO}}\textbf{TODO:} #1}
\newcommand{\note}[1]{\marginpar{#1}}

%%============= THEORMEMS ==================
%\newtheoremstyle{style}
% {\topsep}              % Space above
% {\topsep}              % Space below
% {\itshape}              % Body font: original {\normalfont}
% {}                         % Indent amount (empty = no indent,%\parindent = paraindent)
% {\sffamily\bfseries}  % Thm head font original
% {.}	                     % Punctuation after thm head
% { }                         % Space after thm head (\newline = linebreak)
% {}                          % Thm head spec
%\theoremstyle{style}

%repeating theorems
\newtheorem*{rep@theorem}{\rep@title}
\newcommand{\newreptheorem}[2]{%
\newenvironment{rep#1}[1]{%
 \def\rep@title{#2 \ref{##1}}%
 \begin{rep@theorem}}%
 {\end{rep@theorem}}}

\newreptheorem{lemma}{Lemma'}
\newreptheorem{proposition}{Proposition'}
\newreptheorem{theorem}{Theorem'}

\newtheorem{definition}{Definition}
\newtheorem{theorem}[definition]{Theorem}
\newtheorem{proposition}[definition]{Proposition}
\newtheorem{lemma}[definition]{Lemma}
\newtheorem{corollary}[definition]{Corollary}
\newtheorem{observation}[definition]{Observation}
\newtheorem{conjecture}[definition]{Conjecture}
\newtheorem{remark}[definition]{Remark}
\newtheorem{problem}{Problem}
\newtheorem{claim}{Claim}
\newtheorem{open}{Open Problem}
\newtheorem*{example}{Example}
\newtheorem{assumption}{Assumption}
